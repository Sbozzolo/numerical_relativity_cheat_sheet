\documentclass[]{scrartcl}
\usepackage{amsmath}
\usepackage{amssymb}
\usepackage{siunitx}
\renewcommand{\d}[1]{\ensuremath{\operatorname{d}\!{#1}}}

\DeclareSIUnit[]\sunmass{\text{\ensuremath{M_{\odot}}}}

\author{Gabriele Bozzola}
\title{Numerical Relativity Cheat Sheet}
\date{}
\subtitle{Equations I should remember, but I don't}



\begin{document}
\maketitle
\tableofcontents

\section{Conventions}
\label{sec:conventions}

We denote the metric as $g_{\alpha\beta}$.

\section{ADM Decomposition}
\label{sec:adm-decomposition}

The line element $\d s^2$ is
\begin{equation}
  \label{eq:2}
  \d s^2 = g_{\alpha\beta} \d x^\alpha \d x^\beta = - \alpha^2 \d t^2 + \gamma_{ij} (\d x^i + \beta^i \d t) (\d x^j + \beta^j \d t)
\end{equation}
\begin{equation}
  \label{eq:11}
  \gamma_{a\beta} = g_{\alpha\beta} + n_\alpha n_\beta
\end{equation}
\begin{equation}
  \label{eq:11}
  \gamma^{a\beta} = g^{\alpha\beta} + n^\alpha n^\beta
\end{equation}
\begin{equation}
  \label{eq:12}
  n^\alpha \gamma_{\alpha\beta} = 0
\end{equation}
\begin{equation}
  \label{eq:16}
  t^\alpha = \alpha n^\alpha + \beta^\alpha
\end{equation}
\begin{equation}
  \label{eq:4}
  n^\alpha = \frac{1}{\alpha} (1, - \beta^i)
\end{equation}
\begin{equation}
  \label{eq:5}
  n_\alpha = (-\alpha,0,0,0)
\end{equation}

\subsection{Constraints}
\label{sec:constraints}

Hamiltonian constraint
\begin{equation}
  \label{eq:18}
  {\cal E} = n^\alpha n^\beta T_{\alpha\beta}\,.
\end{equation}

Momentum constraint
\begin{equation}
  \label{eq:9}
  S_i = - \gamma_{i\alpha} n_\beta T^{\alpha\beta}\,.
\end{equation}

\section{Matter and Equations of State}
\label{sec:matter}

Let $\rho_0$ be the rest-mass density, and $\epsilon$ the specific internal energy
density, the total mass-energy $\rho$ measured by an observer comoving with the
fluid is
\begin{equation}
  \label{eq:6}
  \rho = \rho_0 (1 + \epsilon) = \rho_0 + \rho_0 \epsilon = \rho_0 + \varepsilon_{\text{int}}\,,
\end{equation}
with $\varepsilon_{\text{int}}$ internal energy density.

The specific enthalpy is $h = (1 + \epsilon + P\slash \rho_0)$, or $\rho_0 h = \rho + P$.

The Gamma-law equation of state is $P = (\Gamma - 1) \rho_0 \epsilon$, or $P = (\Gamma - 1) \varepsilon_{\text{int}}$.


\subsection{Velocity definitions}
\label{sec:velocity-definitions}

\begin{equation}
  \label{eq:19}
  u_\alpha u^\alpha = -1
\end{equation}

Let $u^\alpha$ be the four-velocity of the fluid.
\begin{equation}
  \label{eq:10}
  W = -n_\alpha u^\alpha
\end{equation}

\begin{equation}
  \label{eq:7}
  u^t = \frac{W}{\alpha}\,.
\end{equation}

\texttt{IllinoisGRMHD}:
\begin{equation}
  \label{eq:13}
  v^i_{\text{IL}} = \frac{u^i}{u^t}
\end{equation}
Valencia:
\begin{equation}
  \label{eq:14}
  v^i_{\text{VA}} = \frac{1}{\alpha} \left( \frac{u^i}{u^t} + \beta^i \right)
\end{equation}
Conversion:
\begin{equation}
  \label{eq:15}
  v^i_{\text{VA}} = \frac{1}{\alpha} \left( v^i_{\text{IL}} + \beta^i \right)
\end{equation}
\begin{equation}
  \label{eq:15}
  v^i_{\text{IL}} = {\alpha} \left( \alpha v^i_{\text{IL}} - \beta^i \right)
\end{equation}
\begin{equation}
  \label{eq:20}
  W = \frac{1}{\sqrt{1 - v^i_{\text{VA}} v_{i,\text{VA}}}}
\end{equation}


Stress-energy tensor of a perfect fluid
\begin{equation}
  \label{eq:8}
  T^{\alpha\beta} = \rho_0 h u^\alpha u^\beta + P g^{\alpha\beta},.
\end{equation}

\section{Global quantities}
\label{sec:global-quantities}

Rest-mass
\begin{equation}
  \label{eq:1}
  M_0 = \int \d{}^{3} x \sqrt{\gamma} W \rho_0
\end{equation}
\section{Useful identities}
\label{sec:useful-identities}

\begin{equation}
  \label{eq:3}
  \sqrt{-g} = \alpha \sqrt{\gamma}
\end{equation}

\section{Numerical values}
\label{sec:numerical-values}

Length-scale associated to \SI{1}{\sunmass}:
\begin{equation}
  \label{eq:length-1-Msun}
  [\textrm{L}] = \frac{GM_{\odot}}{c^{2}} = \SI{1.477}{\km}
\end{equation}
Time-scale associated to \SI{1}{\sunmass}:
\begin{equation}
  \label{eq:length-1-Msun}
  [\textrm{T}] = \frac{GM_{\odot}}{c^{3}} = \SI{4.93}{\micro\s} \approx \SI{5}{\micro\s}
\end{equation}
Frequency-scale associated to \SI{1}{\sunmass}:
\begin{equation}
  \label{eq:length-1-Msun}
  [\textrm{f}] = \frac{1}{[\textrm{T}]} = \frac{c^{3}}{GM_{\odot}} = \SI{203}{\kilo\Hz} \approx \SI{200}{\kilo\Hz}
\end{equation}

\end{document}

%%% Local Variables:
%%% mode: latex
%%% TeX-master: t
%%% End:
